\documentclass[12pt,a4paper]{article}
\usepackage{amsmath}
\usepackage{amsfonts}
\usepackage{amssymb}
\usepackage{amsthm}
\usepackage{framed}

\theoremstyle{definition}
\newtheorem{thm}{قضیه}
\newtheorem{mesal}[thm]{مثال}
\newtheorem{soal}[thm]{سوال}
\newtheorem{tav}[thm]{توجه}
\newtheorem{cor}[thm]{نتیجه}


\newtheorem{tamrinetahvili}{تمرین تحویلی}
\newtheorem{lem}[thm]{لم}

\newtheorem{nokte}[thm]{نکته}
\newtheorem{jambandi}{جمعبندی}
\newtheorem{defn}[thm]{تعریف}

\usepackage{xepersian}
\settextfont{XB Niloofar}
\setdigitfont{XB Niloofar}
\linespread{1.5}
\begin{document}
\section{جلسه‌ی پنجم}
قبلاً ثابت کرده‌ایم که تصویر بردار
$\mathbf{a}$
روی بردار
$\mathbf{b}$
برابر است با
\[
\|a\| \cos \theta \frac{\mathbf{b}}{\|b\|}
\]
همواره هنگام اثبات این فرمول شکلهائی کشیدیم که در آنها زاویه‌ی
$\theta$
میان بردارهای
$\mathbf{a},\mathbf{b}$
کمتر از ۹۰ درجه بود. توجه کنید که اگر زاویه‌ی
یادشده از ۹۰ درجه بیشتر باشد، باز هم همان فرمول کار می‌کند و در این صورت
$\cos \theta$
کمتر از ۰ خواهد شد. در واقع مطابق شکل زیر
تصویر بردار 
$\mathbf{a}$
روی بردار
$\mathbf{-b}$
برابر است با

\[
\|a\| \cos (\pi-\theta) \frac{\mathbf{-b}}{\|b\|}=\|a\| \cos \theta \frac{\mathbf{b}}{\|b\|}
\]
یعنی
تصویر بردار 
$\mathbf{a}$
روی بردار
$\mathbf{-b}$
برابر است با
تصویر بردار 
$\mathbf{a}$
روی بردار
$\mathbf{b}$.
\begin{align*}
\includegraphics[scale=0.15]{riyazi2jal5_1.jpg}
\end{align*}
تا کنون با صفحه‌ها و خطها آشنا شده‌ایم. نیز برخی رویه‌ها را نیز مطالعه کرده‌ایم. هدف بعدی ما در این درس، مطالعه‌ی «منحنی‌های فضائی» است.  برای رسیدن به این هدف، نیازمند 
ابزارهای دیگری در فضاهای برداری هستیم که در این جلسه آنها را مرور می‌کنیم.  تصویرهای زیر هر دو نمونه‌ای از منحنی‌های فضائی هستند. 
\begin{align*}
\includegraphics[scale=0.15]{riyazi2jal5_2.jpg}
\end{align*}
\begin{align*}
\includegraphics[scale=0.15]{riyazi2jal5_3.jpg}
\end{align*}

\section*{ضرب خارجی دو بردار}
با دو بردار می‌توان به یک صفحه رسید. مانند شکل زیر:
\begin{align*}
\includegraphics[scale=0.1]{riyazi2jal5_4.jpg}
\end{align*}
چگونه می‌توان برداری پیدا کرد که بر هر دو بردار 
$\mathbf{a}$
و
$\mathbf{b}$
عمود باشد؟
\\
فرض کنیم 
$\mathbf{a}=(a_1,a_2,a_3)$
و
$\mathbf{b}=(b_1,b_2,b_3)$
دو بردار باشند.
ضرب خارجی 
$\mathbf{a}$
و
$\mathbf{b}$
که آن را با 
$\mathbf{a} \times \mathbf{b}$
نشان می‌دهیم، بردارِ معرفی شده در زیر است:
\[
\mathbf{a} \times \mathbf{b}=(a_2b_3-a_3b_2,a_3b_1-a_1b_3,a_1b_2-a_2b_1)
\]
به بیان دیگر
\[
\mathbf{a} \times \mathbf{b}=(a_2b_3-a_3b_2)\mathbf{i}+(a_3b_1-a_1b_3)\mathbf{j}+(a_1b_2-a_2b_1)\mathbf{k}
\]
\textbf{توجه.}
بردار 
$\mathbf{a} \times \mathbf{b}$
هم بر 
$\mathbf{a}$
و هم بر 
$\mathbf{b}$
عمود است؛ یعنی
\[
(\mathbf{a} \times \mathbf{b})\cdot \mathbf{a} =0
\]
\[
(\mathbf{a} \times \mathbf{b})\cdot \mathbf{b} =0
\]
جهت بردار
$\mathbf{a} \times \mathbf{b}$
با استفاده از قاعده‌ی دست راست تعیین می‌شود. (در کلاس توضیح داده شده است)
\[
\mathbf{a} \times \mathbf{b}=-\mathbf{a} \times \mathbf{b}
\]
برای به خاطر سپردن فرمولِ ضرب خارجی دو بردار، از دترمینانِ یک ماتریس فرضی به صورت زیر استفاده می‌کنیم. 
\[
\mathbf{a}=(a_1,a_2,a_3)
\]
\[
\mathbf{b}=(b_1,b_2,b_3)
\]
\[
A=\left(\begin{array}{ccc}
\mathbf{i} & \mathbf{j} & \mathbf{k} \\
a_1 & a_2 & a_3 \\
b_1 & b_2 & b_3 
\end{array}
\right)
\]
\begin{align*}
&
\mathbf{a} \times \mathbf{b}=\det (A)=
\\
& (a_2b_3-a_3b_2)\mathbf{i}-(a_1b_3-a_3b_1)\mathbf{j}+(a_1b_2-a_2b_1)\mathbf{k}
\end{align*}
\begin{thm}
\begin{enumerate}\hfill
\item
$\mathbf{a} \times \mathbf{b}$
هم بر 
$\mathbf{a}$
و هم بر 
$\mathbf{b}$
عمود است.
\item
ضرب خارجی هر بردار در خودش صفر است.
\[
\mathbf{a} \times \mathbf{a}=0
\]
\[
 (\mathbf{a} \times \mathbf{a})=(a_2a_3-a_3a_2)\mathbf{i}-(a_1a_3-a_3a_1)\mathbf{j}+(a_1a_2-a_2a_1)\mathbf{k}=0
\]
\end{enumerate}
\end{thm}
\begin{align*}
قبلاً ضرب داخلی را تعبیر هندسی کرده‌ بودیم:
\includegraphics[scale=0.15]{riyazi2jal5_5.jpg}
\end{align*}
\[
\mathbf{a} \cdot \mathbf{b}=\|a\| \|b\| \cos \theta
\]
\begin{thm}
اگر 
$\mathbf{a}$
و
$\mathbf{b}$
دو بردار و 
$\theta$
زاویه‌ی بین آن دو باشد، آنگاه 
\[
\|\mathbf{a} \times \mathbf{b} \|=\|\mathbf{a}\| \|\mathbf{b}\| \sin \theta
\]
\begin{align*}
\includegraphics[scale=0.15]{riyazi2jal5_6.jpg}
\end{align*}
بنابراین اندازه‌ی ضرب خارجی 
$\mathbf{a}$
و
$\mathbf{b}$
برابر است با مساخت متوازی‌الاضلاع ساخته شده توسط
$\mathbf{a}$
و
$\mathbf{b}$.
\[
\|\mathbf{a} \times \mathbf{b} \|^2=\|\mathbf{a}\|^2 \|\mathbf{b}\|^2 \overbrace{\sin^2 \theta}^{1-\cos^2 \theta}
\]
اثبات به عهده‌ی دانشجو است.
\end{thm}
\begin{cor}
دو بردار 
$\mathbf{a}$
و
$\mathbf{b}$
با هم موازیند اگر و تنها اگر
\[
\mathbf{a} \times \mathbf{b}=\overrightarrow{0}
\]
\end{cor}
زیرا دو بردار در صورتی موازیند که زاویه‌ی بین آنها یا صفر باشد یا 
$\pi$
و در هر دو صورت
$\sin \theta$
صفر است.
\begin{mesal}
معادله‌ی صفحه‌ای را بیابید که شامل نقاط زیر است.
\begin{align*}
\includegraphics[scale=0.15]{riyazi2jal5_7.jpg}
\end{align*}
\end{mesal}
\begin{proof}[پاسخ]
برای پیدا کردن معادله‌ی صفحه نیازمند دانستن بردار عمود بر آن و نقطه‌ای روی آن صفحه هستیم.
\[
\mathbf{PQ} = (-3,1,-7)
\]
\[
\mathbf{PR} = (0,-5,-5)
\]
می‌دانیم که بردار 
$\mathbf{PQ} \times \mathbf{PR}$
برداری است عمود بر صفحه‌ی مورد نظر ما.
در نتیجه بردار نرمال صفحه برابر است با
\[
\mathbf{PQ} \times \mathbf{PR}=\left|\begin{array}{ccc}
\mathbf{i} & \mathbf{j} & \mathbf{k} \\
-3 & 1 & -7 \\
0 & -5 & -5
\end{array}
\right|
=(-5-35)\mathbf{i}+(-15)\mathbf{j}+(15)\mathbf{k}=(-40,-15,15)
\]
یادآوری می‌کنیم که
معادله‌ی صفحه‌ای که بردار عمودش 
$(a,b,c)$
است و از نقطه‌ی 
$(x_0,y_0,z_0)$
می‌گذرد به صورت زیر است:
\[
ax+by+cz=ax_0+by_0+cz_0
\]
در تنیجه با توجه به بردار نرمال بدست آمده داریم:
\[
-40x-15y+15y=-40 \times 1-15 \times 4+15 \times 6=10
\]
\end{proof}
\begin{mesal}
مساحت مثلث با رئوس زیر را حساب کنید.
\[
P=(1,4,6) , Q=(-2,5,-1), R=(1,-1,1)
\]
\begin{align*}
\includegraphics[scale=0.15]{riyazi2jal5_8.jpg}
\end{align*}
\end{mesal}
\begin{proof}[پاسخ]
قاعده برابر است با اندازه‌ی خط
$PR$
و ارتفاع برابر است با 
$\|PQ\| \sin \theta$
\\
بنابراین مساحت مثلث برابر است با
\[\frac{\|PR\| \|PQ\| \sin \theta}{2}=\frac{\|\mathbf{PR} \times \mathbf{PQ}\|}{2}\]
\end{proof}
\begin{tav}
\[
\mathbf{j}\times \mathbf{k}=\mathbf{i}
\]
\[
\mathbf{k}\times \mathbf{i}=\mathbf{j}
\]
\[
\mathbf{i}\times \mathbf{j}=\mathbf{k}
\]
\begin{align*}
\includegraphics[scale=0.15]{riyazi2jal5_9.jpg}
\end{align*}
\end{tav}
\section*{ویژگی‌های ضرب خارجی}
توجه کنید که در رابطه‌های زیر از آنجا که
$\mathbf{a}\times \mathbf{b}\neq \mathbf{b}\times\mathbf{a}$
باید به ترتیب بردارها دقت داشته باشیم.
\begin{enumerate}
\item
\[
\mathbf{a}\times \mathbf{b}=-\mathbf{b}\times \mathbf{a}
\]
\item
\[
\mathbf{a}\times (\mathbf{b}+\mathbf{c})=\mathbf{a}\times \mathbf{b}+\mathbf{a}\times \mathbf{c}
\]
\[
(\mathbf{b}+\mathbf{c}) \times \mathbf{a}=\mathbf{b}\times \mathbf{a}+\mathbf{c}\times \mathbf{a}
\]
\item
\[
\mathbf{a}\cdot (\mathbf{b} \times \mathbf{c})=(\mathbf{a}\times \mathbf{b}) \cdot \mathbf{c}
\]
\end{enumerate}
\begin{tav}
روش محاسبه‌ی
$\mathbf{a}\cdot (\mathbf{b} \times \mathbf{c})$
به صورت زیر است:
فرض کنید 
$\mathbf{a}=(a_1,a_2,a_3)$
،
$\mathbf{b}=(b_1,b_2,b_3)$
و
$\mathbf{c}=(c_1,c_2,c_3)$
داریم:
\[
\mathbf{a} \cdot (\mathbf{b} \times \mathbf{c})=\left|\begin{array}{ccc}
a_1 & a_2 & a_3 \\
b_1 & b_2 & b_3 \\
c_1 & c_2 & c_3
\end{array}
\right|
\]
\end{tav}
\section*{تعبیر هندسی $\mathbf{c} \cdot (\mathbf{a} \times \mathbf{b})$}
حجم متوازی‌السطوح برابر است با مساحت قاعده در ارتفاع آن.
\begin{align*}
\includegraphics[scale=0.15]{riyazi2jal5_10.jpg}
\end{align*}
مطابق شکل بالا مساحت قاعده برابر است با
$\|\mathbf{a} \times \mathbf{b}\|$.
\\
برای محاسبه‌ی ارتفاع کافی است اندازه‌ی تصویر بردارِ 
$\mathbf{c}$
را روی یک بردار عمود بر 
$\mathbf{a}$
و 
$\mathbf{b}$
محاسبه کنیم.
بردار 
$\mathbf{a} \times \mathbf{b}$
بر 
$\mathbf{a}$
و 
$\mathbf{b}$
عمود است.
پس کافی است تصویر بردار
$\mathbf{c}$
را روی 
$\mathbf{a} \times \mathbf{b}$
پیدا کنیم:
\[
(\|\mathbf{c} \| |\cos \theta|)
\]
پس حجم متوازی‌السطوح یادشده برابر است با:
\[
(\|\mathbf{c} \| |\cos \theta|) (\|\mathbf{a} \times \mathbf{b}\|) \quad (*)
\]
داریم
\[
\cos \theta= \frac{\mathbf{c}.(\mathbf{a}\times\mathbf{b})}{\|\mathbf{c}\|\|\mathbf{a}\times\mathbf{b})\|}
\]
با قرار دادن عبارت بالا در رابطه‌ی
$*$
می‌بینیم که 
حجم متوازی‌السطوح برابر است با
\[|\mathbf{c}.(\mathbf{a}\times \mathbf{b})|\]
\begin{cor}
حجم متوازی‌السطوح ساخته شده توسط بردارهای 
$\mathbf{a}$
،
$\mathbf{b}$
و
$\mathbf{c}$
برابر است با
\[
|\mathbf{c}\cdot (\mathbf{a} \times \mathbf{b})|
\]
\end{cor}
\begin{mesal}
نشان دهید که بردارهای 
$\mathbf{a}=(1,4,-7)$
،
$\mathbf{b}=(2,-1,4)$
و
$\mathbf{a}=(0,-9,18)$
در یک صفحه هستند.
\end{mesal}
\begin{proof}[پاسخ]
کافیست نشان دهیم 
\[
\mathbf{a}\cdot (\mathbf{b} \times \mathbf{c})=0
\]
در واقع اگر این بردارها در یک صفحه نباشند یک متوازی السطوح می‌سازند که حجم آن ناصفر است. محاسبه‌ی عبارت بالا را به عهده‌ی دانشجو می‌گذاریم. 
\end{proof}
\begin{tamrinetahvili}
\hfill
\begin{enumerate}
\item
معادله‌ی خطی را بیابید که از نقاط 
$(4,-1,2)$
و
$(1,1,5)$
می‌گذرد.
\item
معادله‌ی صفحه‌ای را بنویسید که شامل نقاط 
$(3,-1,1)$
،
$(4,0,2)$
و
$(6,3,1)$
است.
\end{enumerate}
\end{tamrinetahvili}
\end{document}